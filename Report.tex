%%%%%%%%%%%%%%%%%%%%%%%%%%%%%%%%%%%%%%%%%%%%%%%%%%%%%%%%%%%%%%%%%%%%%
% Imperial College
\documentclass[a4paper,11pt,twoside]{article}
\usepackage[left=2.5cm,right=2cm,top=2cm,bottom=2cm]{geometry}

%%%%%%%%%%%%%%%%%%%%%%%%%%%%%%%%%%%%%%%%%%%%%%%%%%%%%%%%%%%%%%%%%%%%%
% Paragraph
\usepackage[parfill]{parskip}

% Images
\usepackage{graphicx}

% URLs
\usepackage{hyperref}

% Maths
\usepackage{amsmath}

%%%%%%%%%%%%%%%%%%%%%%%%%%%%%%%%%%%%%%%%%%%%%%%%%%%%%%%%%%%%%%%%%%%%%

\begin{document}

\begin{titlepage}
\begin{center}

Astronomical Image Processing: Galaxy Count from a KPNO Deep Optical 
Image \\[8cm]
Dakshina Scott \\[4cm]
5th November 2012 \\

\end{center}
\end{titlepage}


%%%%%%%%%%%%%%%%%%%%%%%%%%%%%%%%%%%%%%%%%%%%%%%%%%%%%%%%%%%%%%%%%%%%%

\begin{abstract}

Galaxy counts were conducted on a deep optical image taken using the 
4m telescope at KPNO with a Sloan r-band filter. This was done to 
test the relationship between galaxy number counts and magnitude, 
as predicted by the equation \(log_{10} N(m) \propto 0.6m\).


Matlab was used to write a program to detect galaxies within the 
image and catalogue their brightnesses. A plot of galaxy counts 
against magnitude was found to differ significantly from similar 
surveys, and from the above equation. The steepest gradient was 
found at the lower magnitudes as \(0.35 \pm 0.02\) up to magnitude 
13. This is thought to be due to incompleteness in the image rather 
than evidence of strong galaxy evolution or the structure of 
the universe.

\end{abstract}

%%%%%%%%%%%%%%%%%%%%%%%%%%%%%%%%%%%%%%%%%%%%%%%%%%%%%%%%%%%%%%%%%%%%%

\tableofcontents

%%%%%%%%%%%%%%%%%%%%%%%%%%%%%%%%%%%%%%%%%%%%%%%%%%%%%%%%%%%%%%%%%%%%%

\section{Introduction}

Galaxy surveys allow us to probe the history and structure of the 
universe. Number counts and photometric and spectroscopic analysis 
from such surveys have transformed our understanding of the universe - 
when in 1917 Einstein's theory of general relativity suggested that 
the universe must be either expanding or contracting, he brushed the 
idea aside. It was only after Silpher and Hubble produced data from 
galaxy surveys relating redshift and distance that the evidence in 
support of an expanding universe became impossible to ignore\cite{raine}.

In this experiment we looked at a deep optical image taken using a 
Sloan r-band filter (central wavelength 620nm) using the CCD mosaic 
camera at Kitt Peak National Observatory. The aim was to find the 
number counts of galaxies at different magnitudes, and compare our 
results to the prediction for an Euclidean universe of galaxies, 
with no evolution, described by equation~\ref{eq:nmag}. We also 
compared our counts to a similar survey with very high completeness 
in order to evaluate our method and our data.

By assuming an Euclidean universe of uniformly distributed galaxies, 
with a given distribution of luminosities, a simple relationship 
between magnitude and number of galaxies brighter than that magnitude 
can be derived. For a given luminosity distribution
\(f \propto \frac{1}{r^2}\) and so

\begin{equation}
\label{eq:r}
r \propto \frac{1}{f^2}.
\end{equation} 

For a given density in Euclidean space

\begin{equation}
\label{eq:N}
N \propto r^3.
\end{equation}

Substituting equation~\ref{eq:r} into equation~\ref{eq:N} we get

\begin{equation}
N(f) \propto f^{-\frac{3}{2}}
\end{equation}

where N(f) is the number of sources with flux greater than f.

Using

\begin{equation}
m_1 - m_2 = -2.5log_{10}(\frac{f_1}{f_2}),
\end{equation}


the standard equation relating the difference in magnitude 
between two objects and the flux received from them, it can be seen that

\begin{equation}
m \propto -2.5log_{10}(f).
\end{equation}

Rearranging for f

\begin{equation}
\label{eq:f}
f \propto 10^{-2.5m},
\end{equation}

and substituting equation~\ref{eq:f} into equation~\ref{eq:N} 

\begin{equation}
N(m) \propto 10^{0.6m}.
\end{equation}

Finally, taking the log of both sides

\begin{equation}
\label{eq:nmag}
log_{10}(N(m)) \propto 0.6m
\end{equation}

where N(m) is the number of sources brighter than magnitude m.

Based on this result, the difference in the number of galaxies between
each magnitude is given by

\begin{equation}
N(m+1) = 4.0N(m).
\end{equation}

However, the Friedmann equations tell us that the 
universe is not necessarily Euclidean; there may be a distance 
beyond which it's structure influences galaxy counts.

In 1961 Sandage \cite{sandage} found that galaxy counts from the Hale telescope 
(the world's largest telescope at the time\cite{hale}) were unlikely to 
reveal information about the structure of the universe because 
the differences between flat open and closed models are too small compared 
to known variations in galaxy distribution. 
More recently Eisenstein et al (2001)\cite{eisen} also found that the Sloan 
Sky Digital Survey would be well suited to studies of large-scale structure 
by making use of spectroscopic data.

One of the key results looked at in galaxy surveys 
(eg Maddox et al. 1990)\cite{maddox} has been how the data deviates from 
the counts expected by a no-evolution model of galaxies - the further 
away the galaxy, the further in the past we are looking - so if for 
example galaxies were brighter/dimmer or more/less numerous, then 
distant galaxies may give counts inconsistent with equation~\ref{eq:nmag}.

Thus it is  possible that comparing number counts against magnitude 
with the prediction of equation~\ref{eq:nmag} will show evidence of 
galaxy evolution, but it is unlikely that effects of large-scale
structure will be seen.

\section{Method}

Although the image had been processed to adjust for certain effects 
before it was received, there were still some unwanted effects 
remaining. Image processing and analysis was done in Matlab, with 
the image processing toolbox installed.

\subsection{Noise}

The image consists of a combination of a number of sub-exposures. 
This smooths out random noise , as inconsistent 
sources will be faded. It can be seen that these sub-exposures 
don't overlap exactly - as a result there 
are blank areas in the corners of the image, and around the edges a 
sharp increase in noise is visible (see figure~\ref{fig:edges}). 

\begin{figure}[htb]
  \centering
  \includegraphics[width=0.5\textwidth]{edges.jpg}
  \caption{The red lines highlight sharp increases in noise where 
sub-exposures don't overlap. The top right corner is blank.}
  \label{fig:edges}
\end{figure}

The image was cropped before analysis to remove the noisiest areas.

\subsection{Saturated Pixels}

'Blooming' is when pixels in the CCD are saturated and overflow 
to nearby pixels. This  happens mostly in the vertical
 and horizontal directions, as seen in figure~\ref{fig:mask}.

\begin{figure}[htb]
  \centering
  \includegraphics[width=0.5\textwidth]{mask.png}
  \caption{A bright central source which has 'bloomed' vertically across
 the image. There are also smaller sources showing the same effect to 
a lesser extent. Red circles indicate sources which were blacked out 
prior to analysis. The lines of bloom were also blacked out.}
  \label{fig:mask}
\end{figure}

The maximum number of electrons each pixel of the CCD can hold is 65,535. 
For values approaching this, the relationship between number of photons 
received and the number of electrons stored moves further and further 
from linearity. By using multiple shorter exposures any given CCD pixel 
receives fewer photons, so the number of sources bright enough to cause 
overflowing is reduced. However this effect is still visible in some areas 
(highlighted in figure~\ref{fig:mask}). These sources and their bloom were
 blacked out before analysis.

On this fully prepared image, two different algorithms for source 
detection and photometry were applied, each with different limitations. 
One method uses a fixed aperture, while the other uses the shape and size of 
each  source to analyse the correct areas of the image. Both 
methods are dependent upon finding a suitable value for the 
source/background threshold.

\subsection{Finding an Upper Limit for Background Brightness}

In order to analyse each source , some criterion must be used 
to determine how high a photon count must be in order for it to be 
considered to be coming from a source, rather than a background fluctuation.
The image is composed mostly of background, with a scattering of sources 
of varying brightness. Thus the mode pixel value in the 
image gives an estimate of the background value 
However, as the background isn't a constant value, it's clear that 
much of the background will have values higher than this, 
and some upper limit of counts must be chosen as the threshold.
Inspecting the distribution of photon counts within the image shows 
that the frequency of counts per pixel in the region of the modal 
value is many orders of magnitude greater than that for other 
values found in the image, and the distribution within this region 
is approximately Gaussian (see figure~\ref{fig:background}). Thus a 
threshold value was chosen as three standard deviations above the mean 
in this region (a value of 3461 counts).

\begin{figure}[htb]
  \centering
  \includegraphics[width=0.8\textwidth]{background.jpg}
  \caption{Histogram showing the Gaussian-like distribution of 
pixel counts in the region of the global background.} 
  \label{fig:background}
\end{figure}

\subsection{Fixed Aperture Method}

Sources are identified and analysed using a fixed aperture size. 
The maximum pixel value in the image is found, and the surrounding 
area analysed. This area is then blacked out so that a search for 
the new maximum will find the second-brightest source from the 
original image, which is then analysed, and so on. This continues 
looping around until the maximum remaining value equals the threshold value.

A number of different aperture sizes were tried in an attempt to 
find a compromise between a very large and a very small 
aperture. Any source larger than the chosen aperture size ends up 
with a lower count attributed to it than it should have, reduced 
even further when the 'local background' (which in this case will 
at least partially be obscured by the source) is subtracted. Also, 
the smaller the aperture the more opportunity there is for each source 
to be counted as multiple sources (although measures were taken within 
the code to reduce this - see figure~\ref{fig:multisources}). 
Sources which are smaller than the aperture will have a 
reduced mean intensity value, although the total number of counts 
should be about right once the local background has been subtracted. 

\begin{figure}[htb]
  \centering
  \includegraphics[width=0.8\textwidth]{multisources.png}
  \caption{Black circles represent areas that have been treated as 
sources by the detection program. (a) shows those detected before 
code was added telling the program to ignore aperture areas which 
already contained a zero value. (b) shows areas of source detected 
after this section of code was added. As can be seen, there are some 
sources which aren't contained by the aperture.}
  \label{fig:multisources}
\end{figure}

As the aperture size is increased, more than one source is likely 
to be contained within a single aperture. Thus an appropriate size 
would be a compromise. The mean source area is 51 pixels, suggesting 
an aperture radius of 4 pixels (\(pi \times 4^2 \approx 51\))may be a 
reasonable compromise. However, as some of the sources are much larger 
than this, the number counts may be skewed - if lots of apertures 
fit inside one large source, this will give rise to a jump in the 
frequency distribution in the range of brightnesses contained in the 
source.

An aperture of recommended radius 6 pixels\cite{clem} was used as 
a starting point (corresponding to ~3" diameter), centred on the maximum 
value. A larger-radius annular aperture was then placed around 
this to calculate the background level local to each source, which 
was then subtracted from the source aperture value. 
Different radii were breifly compared, including a radius of 4 pixels 
(~2" diameter).

Due to the extensive limitations of the fixed aperture method, a variable 
aperture method was developed and the results looked at in more detail.

\subsection{Variable Aperture Method}

By making better use of inbuilt Matlab functions, the variable aperture is 
much more effective at quickly detecting and analysing sources (it runs in 
about 3 seconds). 

'Thresholding' was used to segment the image based on pixel counts - the 
image was converted to a binary image, in which any pixels with a count 
higher than the source/background threshold (as determined above) become 
ones, the rest zeros. By assigning distinct labels to each separate 
region of ones, Matlab was then able to run through them sequentially, 
analysing each corresponding source and local background as it went.
An issue with this is that sometimes part of the source becomes 'disconnected' 
from the main section (as can be seen in the top left and bottom 
right sides of the binary image in figure~\ref{fig:conversion}).

\begin{figure}[htb]
  \centering
  \includegraphics[width=0.6\textwidth]{conversion.jpg}
  \caption{A section of the image in grayscale, and the same section 
in binary, showing that a good approximation of the source shape and 
area is obtained. However, notice the small white areas near sources 
1 and 2. These are counted as separate sources, regardless of whether 
they are legitimate sources or disconnections from nearby sources.}
  \label{fig:conversion}
\end{figure}

These disconnections may lead to an excess of faint sources being detected.

Another issue arises with finding the local background for each source. 
To do so an area surrounding each source is looked at. If multiple sources 
are close together and their 'local background areas' overlap then the 
same local background, averaged over the regions local to each source, 
is used for all of these sources.

Despite these issues, the variable aperture method was the preferred 
method as it always analyses all of the sources in the image (so 
long as the threshold allows them all to be detected), and unlike 
the fixed aperture it analyses the actual shape and area of each 
source (again limited by the accuracy of the threshold used).

\section{Results \& Analysis}

First the results of the fixed aperture method are briefly looked at, 
with different aperture sizes compared. Then the variable aperture 
method is looked at in more detail and finally the two sets of 
results are briefly compared.
The raw photon counts are converted to magnitudes using 
the equation \(m = ZP_{inst} - 2.5log_{10}(counts)\) 
where \(ZP_{inst}\) is the instrumental zero point, as determined 
in the AB magnitude system (in which 3631 Jy is defined to be 
zero magnitudes). 

According to the fits header for the image 
the maximum reliable value for photon counts per CCD pixel is 
36000 - for a one-pixel source this translates to magnitude 
14. Although we have excluded single-pixel sources from our survey 
this suggests that for magnitude 14 and brighter more sources 
will have unreliable results, as determined by the limitations 
of the equipment used. This should be kept in mind when looking 
at the results.

\subsection{Fixed Aperture}

Number count against magnitude is plotted for various aperture 
radii. The 4-pixel radius appears to produce the straightest 
graph. This could imply that the 'wobbles' in the other graphs 
are due to the limits of a fixed aperture rather than physical phenomena. 

\begin{figure}[htb]
  \centering
  \includegraphics[width=0.9\textwidth]{fixed.png}
  \caption{An aperture radius of 4 pixels has the 'straightest' graph. 
As the radius diverges from this the data points become more and 
more 'wobbly'. In all cases, just below magnitude 14 there was a 
dip in the graph.} 
  \label{fig:fixed}
\end{figure}

\subsection{Variable Aperture}

The plot for a variable aperture shows much more curvature 
than the fixed aperture graphs. This is expected as galaxies 
become gradually more difficult to detect at higher magnitudes, 
so the gradient represents the gradual increase of 
incompleteness in the survey. Excluding those sources which 
contain a pixel value of over 36000 (the maximum good data 
value in the fits header) has very little 
effect - only 14 sources are excluded and the resulting graph 
looks the same.

\begin{figure}[htb]
  \centering
  \includegraphics[width=0.8\textwidth]{varap.png}
  \caption{The errors shown are the errors in the number 
counts of galaxies following a poisson distribution. The 
largest error in the x-axis is 0.023. This is a combination 
of the calibration error on the apparatus used and the error 
in photon counts following a poisson distribution The solid 
line shows a quadratic best fit curve. The dashed line shows 
a gradient of 0.6, the gradient predicted by equation~\ref{nmag}. 
The errors in magnitude aren't included as 
they are too small to show up on the graph 
(see appendix for more details).}
  \label{fig:varap}
\end{figure}


The gradient of the best fit line for magnitudes up to 13 is 
0.34 \(\pm\) 0.02. As can be seen by the curvature of the 
graph this gradually decreases, reaching about 0.1 between 
18 and 20, and it is finally zero (to one decimal place) 
between 21 and 23. 

Even the maximum gradient of about 0.4 at the very start 
of the graph is much lower than the gradient of 0.6 
predicted by equation~\ref{nmag}. If the survey was complete, 
there would be an increase in the galaxy counts at higher magnitudes 
which would increase the gradient. Removing the brightest 
sources (which exhibited a blooming effect) also may have 
affected the gradient; low numbers of bright sources would  
increased the gradient.

Yasuda\cite{yasuda} et al (2001) present galaxy surveys with 
very high levels of completeness, in a variety of bands including 
r-band. At lower magnitudes, where effects of evolution and 
large-scale structure are less significant, their number 
counts fit equation~\ref{nmag} well. This supports the suggestion 
that our data may suffer from incompleteness - and that this 
effect worsens at increasing faintnesses. 

\section{Conclusion}

As discussed the variable aperture is believed to be much 
more reliable in terms of detecting and collecting 
representative data on each source in the image. 
While the program using a fixed aperture gives results that are 
closer to those predicted by theory and found by other surveys, 
it is suspected that this is misleading - that perhaps 
the unquantified systematic errors in the fixed aperture actually 
caused the data to appear closer to the predictions than 
than it truely is. This fits the results found for the variable 
aperture, as it suggests high levels of incompleteness 
starting even at the brighter magnitudes.

Using a larger number of shorter exposures would improve 
the results for the brightest sources as fewer would need 
to be blacked out before analysis. However, this would reduce 
completeness at the fainter magnitudes.
Alternatively using a better CCD (e.g. with a higher bit counter) 
would vastly improve the results - longer exposure times 
could be used allowing fainter sources to be detected more 
reliably, without causing as many pixels to saturate and 
bloom.

Another limitation is that the detecting programs can't distinguish 
between a) stars and galaxies, and b) two overlapping sources. 
Both issues may be addressable by looking at more detailed photometry 
for each source - for example, Yasuda et al (2001)\cite{yasuda} use a 
program called 'PHOTO' which looks for multiple peaks in each object 
and 'deblends' them into multiple 'child' objects. This program is also 
able to determine whether a source is likely to be a star or 
a galaxy based on its surface brightness profile 
(as described by Blanton et al, 2001\cite{blanton}).

Overall, while there is room for improvement, the variable 
aperture program is deemed to be successful at identifying
and analysing sources in the image. However, the incompleteness 
of the image has left us unable to make any reasonable physical 
conclusions regarding magnitude distribution among galaxies.

\appendix

\section{Calculating Errors in Magnitude}

The errors in magnitude are excluded from the graphs as they are very small - and in fact, none are large enough to affect the number of galaxies in each magnitude bin. 
They are approximated by treating the photon counts for each galaxy as a poisson distribution, with the error as the square root of the counts. These errors in counts are then converted into errors in magnitude, and finally added in quadrature with the zeropoint error included in the fits header file, to give the overall errors in magnitude for each galaxy. The largest error was 0.0227. As we are working with magnitude bins of 0.5, it's meaningless to talk about errors to 4 decimal places. To two decimal places, all of the errors are equal to 0.02, the value given for the zeropoint error. 

\begin{figure}[htb]
  \centering
  \includegraphics[width=0.5\textwidth]{errors.png}
  \caption{Distribution of errors with magnitude shows that although 
the errors are greatest for the faintest galaxies, 
all of the errors are very small.}
  \label{fig:varap}
\end{figure}

\begin{thebibliography}{9}

\bibitem{clem}
Clements, D. L., 2012, Computational Image Processing: A Deep Galaxy Survey [Lab script], 2012 ed., Imperial College London

\bibitem{raine}
Raine, D. J., Thomas, E. G., 2002, An Introduction to The Science of Cosmology, 2nd ed., Bristol: Institute of Physics Publishing, p. 60

\bibitem{peacock}
Peacock, J. A., 1999, Cosmological Physics, Cambridge: Cambridge University Press, p. 73

\bibitem{maddox}
Maddox, S. J. et al, 1990, Galaxy Evolution at Low Redshift, Mon. Not. R. astr. Soc., 247, pp.1-5

\bibitem{yasuda}
Yasuda, N., et al. 2001, Galaxy Number Counts From The Sloan Digital Sky Survey Commissioning Data, A.J., 122, pp. 1104-1124

\bibitem{eisen}
Eisenstein, D. J., et al. 2001, Spectroscopic Target Selection For The Sloan Digital Sky Survey: The Luminous Red Galaxy Sample, A.J., 122, pp. 2267-2280

\bibitem{blanton}
Blanton, M. R., et al. 2001, The Luminosity Function Of Galaxies In SDSS Commissioning Data, A.J., 121, pp. 2358-2380

\bibitem{sandage}
Sandage, A., 1961, The Ability of the 200-Inch Telescope to Discriminate Between World Models, Amer. Astr. Soc., 133, pp.335-392

\bibitem{hale}
Caltech Astronomy, 2012. The 200 Inch Hale Telescope. [online] Available at \url{http://www.astro.caltech.edu/palomar/hale.html}
[Accessed 4 November 2012].

\end{thebibliography}


\end{document}

